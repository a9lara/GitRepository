\documentclass[a4paper,11pt]{scrartcl}
\usepackage{geometry}
 \geometry{
 a4paper,
 total={210mm,277mm},
 left=9mm,
 right=9mm,
 top=20mm,
 bottom=30mm,
 }
\usepackage[T1]{fontenc}
\usepackage{multicol}
\usepackage[utf8]{inputenc}
\usepackage{lmodern}
\usepackage{multicol}
\usepackage[english]{babel}
\usepackage{graphicx} 
\usepackage{booktabs} 
\usepackage{float} 
\usepackage{amssymb}
\usepackage{subcaption}
\usepackage{fancyhdr}
\usepackage{zref-user,zref-lastpage}


\fancypagestyle{firstpage}{%
  \fancyhf{}%
  \renewcommand{\headrulewidth}{0mm}%
  \rfoot{hoja \thepage/\zpageref{LastPage}}
} 



\pagestyle{fancy}
\fancyhf{}
\renewcommand{\headrulewidth}{0pt}

\rfoot{hoja \thepage/\zpageref{LastPage}}
\rhead{Nombre (empezando por allidos)}

\begin{document}

\thispagestyle{firstpage}


\begin{flushright}
\begin{tabular}{l@{}}  %% as observed by Werner, use l@{} instead of l
Nombre (empezando por apellidos)\\
Materia\\
Institución \\ 
Nombre del presente documento\\
Prof. \\
Fecha\\
\end{tabular}\end{flushright} 


%%%%%%%%%%% how can I left align the different lines in this flushright ? %%%%%%%%%%%

\bigskip
\bigskip


\textbf{1. Enunciado}\\

Solución\\

\textbf{2. Enunciado}\\

Solución\\

\textbf{3. Enunciado}\\

Solución\\

\newpage

\textbf{4. Enunciado}\\

Solución\\

\textbf{5. Enunciado}\\

Solución\\

\textbf{6. Enunciado}\\

Solución\\


\end{document}
